%\documentclass[handout]{beamer}
\documentclass{beamer}

\usepackage{amsmath,amsfonts,amssymb,graphicx} 

\newcommand\mytitle{An iterated block particle filter for inference on coupled dynamic systems with shared and unit-specific parameters}

\newcommand\ibpf{IBPF}

%% for algorithm pseudocode
\usepackage{enumerate,alltt,xstring}
\usepackage[ruled,noline,linesnumbered]{algorithm2e}
\SetKwFor{For}{for}{do}{end}
\SetKwFor{While}{while}{do}{end}
\SetKwInput{KwIn}{input}
\SetKwInput{KwOut}{output}
\SetKwInput{KwCplx}{complexity}
\SetKwInput{KwIndices}{note}
\SetKwBlock{Begin}{procedure}{}
\DontPrintSemicolon

% for measles model
\newcommand\amplitude{h}
\newcommand\meanBeta{{\bar\beta}}
\newcommand\cohort{c}
\newcommand\birth{b}
\newcommand\gravity{G}

%%%%%%%%%%%%%%%% START OF USER DEFINITIONS %%%%%%%%%%%%%%%5

\newcommand\siOnly[1]{} %% redefined for the supplement
\newcommand\msOnly[1]{#1} %% redefined for the supplement

%% code macros
%\newcommand\slot[1]{\code{#1}}
\newcommand\class[1]{class `\code{#1}'}
\newcommand\code[1]{\texttt{#1}}
\usepackage{url}
\newcommand\spatPomp{\texttt{spatPomp}}

%% EDITING MACROS

%\usepackage[normalem]{ulem}% to use \sout in feedback commands
%\usepackage{subfigure,sidecap}
%\usepackage{wrapfig}
\usepackage{color}
\usepackage[normalem]{ulem}% to use \sout in feedback commands
% orange for EI
\definecolor{orange}{rgb}{1,0.5,0}
\newcommand\ei[2]{\sout{#1} \textcolor{orange}{#2}}
\newcommand\eic[1]{\textcolor{orange}{[#1]}}
% green for KA
\definecolor{green}{rgb}{0,0.5,0}
\newcommand\ka[2]{\sout{#1} \textcolor{green}{#2}}
\newcommand\kac[1]{\textcolor{green}{[#1]}}
%purple for AK
%\definecolor{purple}{rgb}{0.5,0,1}
%\newcommand\ak[2]{\sout{#1} \textcolor{purple}{#2}}
%\newcommand\akc[1]{\textcolor{purple}{[#1]}}
%cyan for JP
\definecolor{cyan}{rgb}{0,.5,.5}
\newcommand\jp[2]{\sout{#1} \textcolor{cyan}{#2}}
\newcommand\jpc[1]{\textcolor{cyan}{[#1]}}

%% basic POMP definitions
\newcommand\Xspace{{\mathbb X}}
\newcommand\Yspace{{\mathbb Y}}
%\newcommand\Thetaspace{{\Theta}}
\newcommand\Thetaspace{\R^{\Thetadim}}
\newcommand\Xunitspace{{\scriptsize{\mathcal X}}}
\newcommand\Yunitspace{{\tiny{\mathcal Y}}}
\newcommand\hatTheta{\widehat{\Theta}}
\newcommand\Xdim{{\mathrm{dim}}(\Xspace)}
\newcommand\Ydim{{\mathrm{dim}}(\Yspace)}
%\newcommand\Thetadim{{\mathrm{dim}}(\Thetaspace)}
\newcommand\Thetadim{P}
\newcommand\thetadim{p}
\newcommand\timeSet{{\mathbb T}}
\newcommand\data[1]{#1^*}

\newcommand\unitSpecific{\hspace{0.1mm}\mathrm{us}}
\newcommand\shared{\hspace{0.15mm}\mathrm{sh}}

\newcommand\comma{{\hspace{-0.25mm},\hspace{-0.2mm}}}

%% for particle filters
\newcommand\unit{u}
\newcommand\altUnit{\tilde u}
\newcommand\Unit{U}
\newcommand\rep{i}
\newcommand\Rep{\mathcal{I}}
\newcommand\island{\rep}
\newcommand\Island{\Rep}
%\newcommand\txtisland{replicate}
\newcommand\altIsland{\tilde i}
\usepackage[mathscr]{euscript}

\renewcommand\time{n}
\newcommand\myvec[1]{\boldsymbol{#1}}
\newcommand\altTime{\tilde n}
\newcommand\Time{N}
\newcommand\Np{J}
\newcommand\np{j}
\newcommand\altNp{k}
\newcommand\altAltNp{\tilde j}
\newcommand\resampleIndex{i}
\newcommand\unitWeight{w}
\newcommand\spatReg{r}


%% for all iterated filtering algorithms
\newcommand\Nit{M}
\newcommand\nit{m}

%% for Bpfilter	
\newcommand\block{k}
\newcommand\Block{K}
\newcommand\blockweight{w}
\newcommand\blocklist{\mathcal{B}} %% Partition

%% customized math macros
\newcommand\prob{\mathbb{P}}
\newcommand\E{\mathbb{E}}
\newcommand\dd[1]{\mathrm{d}{#1}}
\newcommand\given{{\,\vert\,}}
\newcommand\equals{{{\,}={\,}}} 
\newcommand\myequals{\hspace{0.5mm}{=}\hspace{0.5mm}}
\newcommand\myto{{\;:\;}}
\newcommand\mycolon{{\hspace{0.6mm}:\hspace{0.6mm}}}
\newcommand\seq[2]{{#1}\mycolon{#2}}
\newcommand\mydot{{\,\cdot\,}}
\newcommand\cp[2]{N_{\mathrm{#1}\mathrm{#2}}}
\newcommand\giventh{{\hspace{0.5mm};\hspace{0.5mm}}}
%\newcommand\normal{{\mathrm{Normal}}}
\newcommand\normal{\mathcal{N}}
\newcommand\argequals{{\,=\,}}
\newcommand\lags{c}
\newcommand\maxlag{\overline{c}}
\newcommand\nlfList{C}
\newcommand\bigO{\mathcal{O}}
\newcommand\loglik{\lambda}
\newcommand\loglikMC{\MC{\loglik}}
\newcommand\R{\mathbb{R}}
\newcommand\param{\,;}
\newcommand\MC[1]{#1^{\,\mbox{\tiny MC}}}
\newcommand\EMC{{\E}}
\newcommand\Var{\mathrm{Var}}
\newcommand\var{\Var}
\newcommand\Cov{\mathrm{Cov}} 
\newcommand\cov{\Cov}
\newcommand\iid{\mathrm{iid}}
%\newcommand\dist{\mathrm{dist}}
\newcommand\dist{d}
\newcommand\transpose{\top}

\def\lik{L}
\def\loglik{\ell}

%% THEOREM-LIKE ENVIRONMENTS
\usepackage{amsthm}
\newtheorem{prop}{Proposition}
\newtheorem{theorem}{Theorem}
\newtheorem{lemma}{Lemma}
\newtheorem{example}{Example}
\newtheorem{remark}{Remark}
\newtheorem{cor}{Corrolary}
\newtheorem{fact}{Fact}  
\newtheorem{assumption}{Assumption}
\newtheorem{assumptionB}{Assumption}
%\newtheorem{procedure}{Procedure}
\renewcommand\theassumption{A\arabic{assumption}}
\renewcommand\theassumptionB{B\arabic{assumptionB}}

%\usepackage{float}
%\floatstyle{ruled}
%\newfloat{textbox}{t}{tbx}
%\floatname{textbox}{Box}
%\renewcommand\thetextbox{\arabic{textbox}}



%%%%%%%%%%%%%%%% END OF USER DEFINITIONSS %%%%%%%%%%%%%%%5

\newcommand\myemph[1]{{\textbf{#1}}}
\newcommand\enumerateSpace{\hspace{2mm}}
\usepackage{amssymb}
\newenvironment {myitemize} {
                 \begin{list}{\textcolor{black}{$\bullet$} \hfill}
%                 \begin{list}{\textcolor{blue}{{\small{$\blacktriangleright$}}} \hfill}
                 {\setlength{\labelwidth}{0.3 cm}
                  %\setlength{\leftmargin}{0em}
                  \setlength{\leftmargin}{0.15cm}
                  \setlength{\itemindent}{0.15cm}
                  \setlength{\labelsep}{0cm}
                  \setlength{\parsep}{0.2 ex}
%                  \setlength{\itemsep}{0.25 cm}
                  \setlength{\itemsep}{1 mm}
%                   \setlength{\itemsep}{0.0 cm}
      \setlength{\topsep}{0.0cm}}} %space between title and 1st item
   {\end{list}}

\newcommand\MainFigureBmScaling{1}
\newcommand\MainFigureMeaslesScaling{2}
\newcommand\MainFigureMeaslesSlice{3}

%% for algorithm pseudocode
\usepackage{enumerate,alltt,xstring}
%\usepackage[ruled,noline,linesnumbered]{algorithm2e}
\usepackage[ruled,noline]{algorithm2e}
\SetKwFor{For}{for}{do}{end}
\SetKwFor{While}{while}{do}{end}
\SetKwInput{KwIn}{input}
\SetKwInput{KwOut}{output}
\SetKwInput{KwCplx}{complexity}
\SetKwInput{KwIndices}{note}
\SetKwBlock{Begin}{procedure}{}
\DontPrintSemicolon

%\newcommand\SuppFigureMeaslesSharedSlice{\eic{TODO}}

\newcommand\SuppSecGeneralization{S1}
\newcommand\SuppSecThmI{S3}
\newcommand\SuppSecThmII{S4}
\newcommand\SuppSecMemoryEfficient{S10}
\newcommand\SuppSecAdaptedSimulation{S2}
\newcommand\SuppSecBM{S5}
\newcommand\SuppSecMeasles{S6}
\newcommand\SuppSecMeaslesNbhd{S7}
\newcommand\SuppSecMeaslesGammaSlice{S8}
%\newcommand\SuppSecMeaslesReducedNoise{S9}
\newcommand\SuppSecLorenz{S9}
%\newcommand\SuppSecMeaslesIJ{S11}

%\newcommand\AssumptionBivb{\ref{B:unconditional:mix}(b)}
%\newcommand\AssumptionBiva{\ref{B:unconditional:mix}(a)}
\newcommand\comma{{\hspace{-0.25mm},\hspace{-0.2mm}}}
\newcommand\termStart{\dagger}
%\newcommand\termStart{\circ}

\newcommand\amplitude{a}
\newcommand\meanBeta{{\bar\beta}}
\newcommand\cohort{c}
\newcommand\birth{b}


%\title{Locally weighted island filters for partially observed spatiotemporal systems}

\title{Island filters for partially observed spatiotemporal systems}


%%%%%%%%%%%%%%%% START OF USER DEFINITIONS %%%%%%%%%%%%%%%5

\newcommand\siOnly[1]{} %% redefined for the supplement
\newcommand\msOnly[1]{#1} %% redefined for the supplement

%\newcommand\tif{\mathrm{tif}}
\newcommand\inputSpace{\rule[-1.5mm]{0mm}{4.7mm}}
\newcommand\firstLineSpace{\rule[0mm]{0mm}{4mm}}
\newcommand\lastLineSpace{\rule[-2.5mm]{0mm}{4mm}}

\newcommand\gravity{G}

\newcommand\AIRSIF{ASIF-IR}
\newcommand\TIF{BIF}
\newcommand\centersum{\Delta}
\newcommand\mediumint{\resizebox{2.8mm}{!}{$\displaystyle \int$}}
%\newcommand\Bsize{|B|}
\newcommand\Bsize{b}
%\newcommand\UnitSet{\mathbb{U}}
%\newcommand\TimeSet{\mathbb{N}}
\newcommand\TimeSet{\mbox{TimeSet should now be ObsTimeSet or ProcTimeSet}}
\newcommand\UnitSet{1{\mycolon}\Unit}
\newcommand\ObsTimeSet{1{\mycolon}\Time}
\newcommand\ProcTimeSet{0{\mycolon}\Time}
%\newcommand\spaceTime{\mathbb{S}}
\newcommand\spaceTime{\UnitSet\times\ObsTimeSet}
%\newcommand\unitTimeSubset{\mathbb{C}}
\newcommand\unitTimeSubset{B}
%\newcommand\adapted{\rho}
\newcommand\adapted{\gamma}
\newcommand\myeqref[1]{[\ref{#1}]}
\newcommand\tif{{}}
\newcommand\girf{\mathrm{girf}}
\newcommand\gir{G}
\newcommand\altAltTime{m}

\newcommand\myvec[1]{\boldsymbol{#1}}

\newcounter{wwii}
\setcounter{wwii}{1}
\newcommand\wwiiStep{\makebox[8mm][l]{\arabic{wwii}.}\stepcounter{wwii}}
\newcounter{wwiiResample}
\newcounter{wwiiAncestor}
\newcounter{wwiiPredict}
\newcounter{wwiiFilter}

\newcounter{asii}
\setcounter{asii}{1}
\newcommand\asiiStep{\makebox[8mm][l]{\arabic{asii}.}\stepcounter{asii}}
\newcounter{girasif}
\setcounter{girasif}{1}
\newcommand\girasifStep{\makebox[8mm][l]{\arabic{girasif}.}\stepcounter{girasif}}
\newcounter{asiiResample}
\newcounter{asiiAncestor}
\newcounter{asiiPredict}
\newcounter{asiiFilter}
\newcounter{girasifResample}
\newcounter{girasifAncestor}
\newcounter{girasifPredict}
\newcounter{girasifFilter}


% package used by default for Overleaf
%\usepackage[utf8]{inputenc}

%% code macros
%\newcommand\slot[1]{\code{#1}}
%\newcommand\class[1]{class `\code{#1}'}
\newcommand\code[1]{\texttt{#1}}
\usepackage{url}

\newcommand\W{W}
\newcommand\V{V}
\newcommand\ASindex{[\Unit]}
\newcommand\simulation{\mathrm{sim}}

%\renewcommand\baselinestretch{1.3}
%\usepackage{showkeys}
%\usepackage[numbers,sort&compress]{natbib}
%\bibliographystyle{apalike}

%\usepackage{natbib}
%\bibliographystyle{plainnat}
%\bibliographystyle{dcu}
%\usepackage{url}

\newcommand\pstep[1]{{\bf #1}}
\newcommand\qsp{\rule{0mm}{1mm}\hspace{4mm}}
\newcommand\plusStrut{\rule{0mm}{1.2mm}}

%% basic POMP definitions
\newcommand\Xspace{{\mathbb X}}
\newcommand\Yspace{{\mathbb Y}}
%\newcommand\Thetaspace{{\Theta}}
\newcommand\Thetaspace{\R^{\Thetadim}}
\newcommand\Xunitspace{{\scriptsize{\mathcal X}}}
\newcommand\Yunitspace{{\tiny{\mathcal Y}}}
\newcommand\hatTheta{\widehat{\Theta}}
\newcommand\Xdim{{\mathrm{dim}}(\Xspace)}
\newcommand\Ydim{{\mathrm{dim}}(\Yspace)}
%\newcommand\Thetadim{{\mathrm{dim}}(\Thetaspace)}
\newcommand\Thetadim{P}
\newcommand\thetadim{p}
\newcommand\timeSet{{\mathbb T}}
\newcommand\data[1]{#1^*}

\newcommand\unitSpecific{\hspace{0.1mm}\mathrm{us}}
\newcommand\shared{\hspace{0.15mm}\mathrm{sh}}

%% for particle filters
\newcommand\unit{u}
\newcommand\altUnit{\tilde u}
\newcommand\Unit{U}
\newcommand\rep{i}
\newcommand\altIsland{\tilde i}
\usepackage[mathscr]{euscript}
\newcommand\Rep{\mathcal{I}}
\renewcommand\time{n}
\renewcommand\vec[1]{\boldsymbol{#1}}
\newcommand\altTime{\tilde n}
\newcommand\Time{N}
\newcommand\Np{J}
\newcommand\np{j}
\newcommand\altNp{k}
\newcommand\altAltNp{\tilde j}
\newcommand\resampleIndex{r}
\newcommand\unitWeight{w}


%% for ASIF and HIPPIE
\newcommand\IF{\mathrm{A}}  % adapted simulation
\newcommand\IP{\mathrm{P}}  % adapted proposal
%\newcommand\GR{\mathrm{GR}}  % GIR resample 
%\newcommand\GP{\mathrm{GP}}  % GIR proposal 
\newcommand\GR{\mathrm{IR}}  % intermediate resample 
\newcommand\GP{\mathrm{IP}}  % intermediate proposal 
\newcommand\resample{\mathrm{R}} % this would have been F before
\newcommand\proposal{\mathrm{P}} 
\newcommand\LCP{\mathrm{P}}  % local prediction 
\newcommand\LCF{\mathrm{F}}  % local filter 

%% for GIR
%\newcommand\Lookahead{L}
%\newcommand\lookahead{\ell}
\newcommand\Ninter{S} %% number of intermediate timepoints
\newcommand\ninter{s}
%\newcommand\Nguide{K} %% number of lookahead particles
%\newcommand\nguide{k}
%\newcommand\lookaheadEnd{\time\oplus \Lookahead}
%\newcommand\lookaheadEnd{\min(\time+\Lookahead,\Time)}
\newcommand\guideFunc{g}
%\newcommand\VtoTheta{{\overleftarrow{V}}}
%\newcommand\thetaToV{{\overrightarrow{V}}}
%\newcommand\VtoTheta{{\overline{V}}}
%\newcommand\thetaToV{\underline{V}}
%\newcommand\thetaToV[1]{\stackrel{\rightarrow}{\mathrm{v}}_{\hspace{-0.5mm}#1}\!}
%\newcommand\VtoTheta[1]{\stackrel{\leftarrow}{\mathrm{v}}_{\hspace{-0.5mm}#1}\!}
\newcommand\thetaToV{\stackrel{\rightarrow}{\mathrm{v}}\!}
\newcommand\VtoTheta{\stackrel{\leftarrow}{\mathrm{v}}\!}
%\newcommand\thetaToV[1]{\stackrel{\rightarrow}{\mathrm{v}_{#1}}\!}
%\newcommand\VtoTheta[1]{\stackrel{\leftarrow}{\mathrm{v}_{#1}}\!}
\newcommand\npgir{\np}
\newcommand\Npgir{\Np}
\newcommand\Nguide{G}

%% for all iterated filtering algorithms
\newcommand\Nit{M}
\newcommand\nit{m}


%% customized math macros
\newcommand\prob{\mathbb{P}}
\newcommand\E{\mathbb{E}}
\newcommand\dd[1]{\mathrm{d}{#1}}
\newcommand\given{{\,\vert\,}}
\newcommand\equals{{{\,}={\,}}} 
\newcommand\myequals{\hspace{0.5mm}{=}\hspace{0.5mm}}
\newcommand\myto{{\;:\;}}
\newcommand\seq[2]{{#1}\!:\!{#2}}
\newcommand\mydot{{\,\cdot\,}}
\newcommand\cp[2]{N_{\mathrm{#1}\mathrm{#2}}}
\newcommand\giventh{{\hspace{0.5mm};\hspace{0.5mm}}}
%\newcommand\normal{{\mathrm{Normal}}}
\newcommand\normal{\mathcal{N}}
\newcommand\argequals{{\,=\,}}
\newcommand\lags{c}
\newcommand\maxlag{\overline{c}}
\newcommand\nlfList{C}
\newcommand\bigO{\mathcal{O}}
\newcommand\loglik{\lambda}
\newcommand\loglikMC{\MC{\loglik}}
\newcommand\R{\mathbb{R}}
\newcommand\param{\,;}
\newcommand\mycolon{{\hspace{0.6mm}:\hspace{0.6mm}}}
\newcommand\MC[1]{#1^{\,\mbox{\tiny MC}}}
%\newcommand\EMC{\MC{\E}}
\newcommand\EMC{{\E}}
\newcommand\Var{\mathrm{Var}}
\newcommand\var{\Var}
\newcommand\Cov{\mathrm{Cov}} 
\newcommand\cov{\Cov}
\newcommand\iid{\mathrm{iid}}
%\newcommand\dist{\mathrm{dist}}
\newcommand\dist{d}
\newcommand\transpose{\top}

\newcommand\gnsep{,}

\def\lik{L}
\def\loglik{\ell}

%% THEOREM-LIKE ENVIRONMENTS
\usepackage{amsthm}
\newtheorem{prop}{Proposition}
%\newtheorem{theorem}{Theorem}
%\newtheorem{lemma}{Lemma}
%\newtheorem{example}{Example}
\newtheorem{remark}{Remark}
\newtheorem{cor}{Corrolary}
%\newtheorem{fact}{Fact}  
\newtheorem{assumption}{Assumption}
\newtheorem{assumptionB}{Assumption}
%\newtheorem{procedure}{Procedure}
\renewcommand\theassumption{A\arabic{assumption}}
\renewcommand\theassumptionB{B\arabic{assumptionB}}

%\usepackage{float}
%\floatstyle{ruled}
%\newfloat{textbox}{t}{tbx}
%\floatname{textbox}{Box}
%\renewcommand\thetextbox{\arabic{textbox}}


%% FOR PSEUDOCODE SETUP
\newcommand\mystretch{\rule[-2mm]{0mm}{5mm} }   
\newcommand\asp{\hspace{4mm}}

%%%%%%%%%%%%%%%% END OF USER DEFINITIONSS %%%%%%%%%%%%%%%5


\newcommand\PartOfBoundOffDiagonal{3\efourB+4(K + \dmax)\,\efive}
\newcommand\boundOffDiagonal{\efourA + \PartOfBoundOffDiagonal}
\newcommand\K{K^\prime}
\newcommand\ThmOneVarBound{Q^{4b} \, \Unit\Time  \big(\altb + \etwo \, (\Unit\Time-\altb) \big)}


\newcommand\BvConstant{\mathcal{C}_0}
\newcommand\dmax{d_{\max}}
\newcommand\boundLemma{\efourB+(K+d_{\unit,\time})\efive}
\newcommand\boundLemmaUniform{\efourB+(K+\dmax)\efive}


%\newcommand\altB{\tilde{B}}
\newcommand\altB{C}
%\newcommand\el{\epsilon_\ell}
\newcommand\el{\epsilon}   %%% epsilon for Thm 1
%\newcommand\eone{\epsilon_1}
%\newcommand\eone{\epsilon^{}_{\!\mathrm{\ref{A1}}}}
\newcommand\eone{\epsilon^{}_{\!\mathrm{A1}}}
%\newcommand\eone{\epsilon^{A}_{1}}
%\newcommand\etwo{\epsilon^{}_{\!\mathrm{\ref{A:unconditional:mix}}}}
\newcommand\etwo{\epsilon^{}_{\!\mathrm{A4}}}
%\newcommand\altb{\tilde{b}}
\newcommand\altb{c}
\newcommand\Ai{
\begin{assumption}\label{A1}
There is an $\eone>0$ and a collection of neighborhoods $\{B_{\unit\comma\time}\subset A_{\unit\comma\time}, \unit\in 1\mycolon\Unit,\time\in 1\mycolon\Time\}$ such that, for all  $\unit$, $\time$, $\Unit$, $\Time$,  any  bounded real-valued function $|h(x)|\le 1$, and any value of $x_{B^c_{\unit\comma\time}}$, 
%\begin{equation}
\begin{eqnarray}
\nonumber
&& \hspace{-12mm} \Bigg| \int h(x_{\unit\comma\time}) f_{X_{\unit\comma\time}|Y_{B_{\unit\comma\time}},X_{B^c_{\unit\comma\time}}}(x_{\unit\comma\time}\given  \data{y}_{B_{\unit\comma\time}}, x_{B^c_{\unit\comma\time}}) \, dx_{\unit\comma\time}
\\
\siOnly{\label{eq:weak_coupling2}}
\msOnly{\nonumber}
&& \hspace{-3mm} - \int h(x_{\unit\comma\time})f_{X_{\unit\comma\time}|Y_{B_{\unit\comma\time}}}(x_{\unit\comma\time}\given \data{y}_{B_{\unit\comma\time}})
\, dx_{\unit\comma\time}
\Bigg| \hspace{1mm} < \hspace{1mm} \eone.
%\end{equation}
\end{eqnarray}
\end{assumption}
}

\newcommand\Aii{
\begin{assumption}\label{A1b}
For the collection of neighborhoods in Assumption~\ref{A1}, with $B^+_{\unit\comma\time}=B^{}_{\unit\comma\time}\cup (\unit,\time)$, there is a constant $\Bsize$, depending on $\eone$ but not on $\Unit$ and $\Time$, such that
\begin{equation}
\nonumber
\sup_{\unit\in 1:\Unit, \, \time\in 1:\Time} \big| B^+_{\unit\comma\time}\big| \le \Bsize.
\end{equation}
\end{assumption}
}

\newcommand\Aiii{
\begin{assumption}\label{A2}
There is a constant $Q$ such that, for all $\unit$ and $\time$, $\Unit$ and $\Time$,
\begin{equation}
\nonumber
Q^{-1} <f_{Y_{\unit\comma\time}|X_{\unit\comma\time}}(\data{y}_{\unit\comma\time}\given x_{\unit\comma\time}) < Q
\end{equation}
\end{assumption}
}

\newcommand\Aiv{
  \begin{assumption}\label{A:unconditional:mix}
    There exists $\etwo > 0$ such that the following holds.
    For each $\unit, \time$, a set $\altB^{}_{\unit\comma\time} \subset (1\mycolon U)\times(0\mycolon N)$ exists such that $(\altUnit, \altTime) \notin \altB_{\unit\comma\time}$ implies $B_{\unit\comma\time}^+ \cap B_{\altUnit\comma\altTime}^+ = \emptyset$ and
      \begin{equation} 
\siOnly{\label{eq:A:unconditional:mix}}
\msOnly{\nonumber}
        \big| f_{X_{B^{+}_{\altUnit\comma\altTime}}|X^{}_{B^{+}_{\unit\comma\time}}} - f_{X_{B^{+}_{\altUnit\comma\altTime}}} \big|
        < \etwo \, f_{X_{B^{+}_{\altUnit\comma\altTime}}}
      \end{equation}
      Further, there is a uniform bound $|\altB_{\unit\comma\time}| \le \altb$.
\end{assumption}
}

\newcommand\TheoremI{
\begin{theorem}  \label{thm:tif}
Let $\MC{\loglik}$ denote the Monte Carlo likelihood approximation constructed by {\TIF}.
Consider a limit with a growing number of islands, $\Island\to\infty$, and suppose assumptions~\ref{A1},~\ref{A1b} and~\ref{A2}.
There are quantities $\el(\Unit,\Time)$ and $V(\Unit,\Time)$, with bounds $|\el|<\eone Q^{2}$ and $V < Q^{4\Bsize} \, U^2 N^2$, such that
\begin{equation}
\label{th1:lik:bound}
\Island^{1/2}\big[ \MC{\loglik}-\loglik-\el\Unit\Time \big]  \xrightarrow[\Island \rightarrow \infty]{d} \normal\big[0,V\big],
\end{equation}
where $\xrightarrow[\Island \rightarrow \infty]{d}$ denotes convergence in distribution and $\normal[\mu,\Sigma]$ is the normal distribution with mean $\mu$ and variance $\Sigma$.
If additionally Assumption~\ref{A:unconditional:mix} holds, we obtain an improved variance bound
\begin{equation}
\label{th1:lik:bound2}
V < \ThmOneVarBound.
\end{equation}
\end{theorem}
}


%%% 2222222222222222222222


%\newcommand\ethree{\epsilon^{}_{\mathrm{\ref{B1}}}}
\newcommand\ethree{\epsilon^{}_{\mathrm{B1}}}
%\newcommand\efourA{\epsilon^{}_{\mathrm{\ref{B:unconditional:mix}}}}
\newcommand\efourA{\epsilon^{}_{\mathrm{B4}}}
%\newcommand\efourB{\epsilon^{}_{\mathrm{\ref{B:temporal:mix}}}}
\newcommand\efourB{\epsilon^{}_{\mathrm{B5}}}
%\newcommand\efive{\epsilon^{}_{\mathrm{\ref{B:girf}}}}
\newcommand\efive{\epsilon_{\mathrm{B6}}}

\newcommand\Bi{

\begin{assumptionB}\label{B1}
  There is an $\ethree>0$ and a collection of neighborhoods $\{B_{\unit\comma\time}\subset A_{\unit\comma\time}, \unit\in 1\mycolon\Unit,\time\in 1\mycolon\Time\}$ such that the following holds for all $\unit$, $\time$, $\Unit$, $\Time$, and any bounded real-valued function $|h(x)|\le 1$\emph{:} if we write $A=A_{u,n}$, $B=B_{u,n}$, $f_{A}(x_A) = f_{Y_{A}|X_{A}}(\data y_{A}|x_A)$, and $f_{B}(x_B) = f_{Y_{B}|X_{B}}(\data y_{B}|x_B)$,
%\begin{equation}
\begin{eqnarray}
\nonumber
%\label{asif:eq:weak_coupling2}
&&\hspace{-7mm}
\left| 
\int \hspace{-1mm} h(x)
\left\{
\frac{
  \E_{g} \! \big[ 
    f_{A}\big(X_{A}^P\big) f_{X_{u,n}|X_{A^{[n]}},\vec X_{n-1}} \big(x | X_{A^{[n]}}^P, \vec X_{n-1}\big) 
  \big]
} {
  \E_{g} \! \big[f_A\big(X_A^P\big)\big]
} - 
\right.
\right.
\\
\nonumber
&& 
\siOnly{\hspace{20mm}}
\hspace{-6.5mm} 
% - \hspace{-1mm} 
\siOnly{\hspace{2mm}}
\left.
\left.
\frac{
  \E_{g} \! \big[
    f_{B}\big(X_{B}^P\big) f_{X_{u,n}|X_{B^{[n]}},\vec X_{n-1}}\big(x | X_{B^{[n]}}^P, \vec X_{n-1}\big) \big] 
} {
  \E_{g} \! \big[f_B\big(X_B^P\big)\big]
} 
\!
\right\}
\!
dx
\right| \hspace{0mm} < \hspace{0mm} \ethree.
%\end{equation}
\end{eqnarray}
\end{assumptionB}
}

\newcommand\Bii{
\begin{assumptionB}\label{B1b}
The bound $\sup_{\unit\in 1:\Unit,\time\in 1:\Time} \big| B^+_{\unit\comma\time}\big| \le \Bsize$ in Assumption~\ref{A1b} applies for the neighborhoods defined in Assumption~\ref{B1}. 
This also implies there is a finite maximum temporal depth for the collection of neighborhoods, defined as
\begin{equation}
\nonumber
\dmax=\sup_{(\unit,\time)}\hspace{1mm} \sup_{(\altUnit,\altTime)\in B_{\unit,\time}}|\time-\altTime|.
\end{equation}
\end{assumptionB}
}

\newcommand\Biii{
\begin{assumptionB}\label{B2} Identically to Assumption~\ref{A2}, 
$Q^{-1} <f_{Y_{\unit\comma\time}|X_{\unit\comma\time}}(\data{y}_{\unit\comma\time}\given x_{\unit\comma\time}) < Q$.
\end{assumptionB}
}

\newcommand\BivA{
\begin{assumptionB}\label{B:unconditional:mix}
Define $g_{{X}_{A}|{X}_{B}}$ via \myeqref{eq:g}.
Suppose there is an $\efourA$ such that the following holds.
For each $\unit, \time$, a set $\altB^{}_{\unit\comma\time} \subset (1\mycolon U)\times(0\mycolon N)$ exists such that $(\altUnit, \altTime) \notin \altB_{\unit\comma\time}$ implies $B_{\unit\comma\time}^+ \cap B_{\altUnit\comma\altTime}^+ = \emptyset$ and
%      \begin{eqnarray} 
%\label{eq:B:unconditional:mix:1}
\begin{equation}
\nonumber
\big| 
   g^{}_{{X}^{P}_{B^{}_{\altUnit\comma\altTime}}{X}^{P}_{B^{}_{\unit\comma\time}}} 
   - g^{}_{{X}^{P}_{B^{}_{\altUnit\comma\altTime}}} \, g^{}_{{X}^{P}_{B^{}_{\unit\comma\time}}} 
\big|
<
%&\hspace{-3mm}<\hspace{-3mm} & 
%\\
%&&\hspace{-30mm}
(1/2) \, \efourA \, \,  g^{}_{{X}^{P}_{B^{}_{\altUnit\comma\altTime}}} \, g^{}_{{X}^{P}_{B^{}_{\unit\comma\time}}}
%\\
%\label{eq:B:unconditional:mix:2}
\end{equation}
\begin{eqnarray}
\nonumber
\big|
g^{}_{X^P_{B^{}_{\altUnit\comma\altTime}}|\vec{X}_{0:\Time}} 
\, g^{}_{X^P_{B^{}_{\unit\comma\time}}|\vec{X}_{0:\Time}} 
-
g^{}_{X^P_{B^{}_{\altUnit\comma\altTime}},X^P_{B^{}_{\unit\comma\time}}|\vec{X}_{0:\Time}}
\big|
&&
%&\hspace{-3mm}<\hspace{-3mm}& 
\\
\nonumber
%&&\hspace{-30mm}(1/2) \, \efourA \, \, 
&&\hspace{-30mm} < (1/2) \, \efourA \, \, 
 g^{}_{X^P_{B^{}_{\altUnit\comma\altTime}},X^P_{B^{}_{\unit\comma\time}}|\vec{X}_{0:\Time}}
      \end{eqnarray}
      Further, there is a uniform bound $|\altB_{\unit\comma\time}| \le \altb$.
\end{assumptionB}
}
\newcommand\BivB{
\begin{assumptionB}\label{B:temporal:mix}
There is a constant $K$ such that, for any set $D\subset (\UnitSet)\times(n{\mycolon}n-d)$ and for all $\time\ge K+d$, $\vec{x}^{(1)}_{\time-d-K}$, $\vec{x}^{(2)}_{\time-d-K}$, and all $x^{}_{D_n}$,
\begin{eqnarray}
\nonumber
&&\hspace{-6mm} \big|
g^{}_{{X}_{D}|\vec{X}_{\time-d-K}}(x^{}_{D}\given\vec{x}^{(1)}_{\time-d-K})
-
g^{}_{{X}_{D}|\vec{X}_{\time-d-K}}(x^{}_{D}\given\vec{x}^{(2)}_{\time-d-K})
\big|
\\
\nonumber
&& \siOnly{\hspace{10mm}} \hspace{4mm}
< 
\efourB  \,  \,
g^{}_{{X}_{D}|\vec{X}_{\time-d-K}}(x^{}_{D_n}\given\vec{x}^{(1)}_{\time-d-K})
\end{eqnarray}

\end{assumptionB}
}

\newcommand\Bv{
\begin{assumptionB}\label{B:girf}
Let $h$ be a bounded function with $|h(x)|\le 1$. 
Let $\vec{X}^{\GR}_{\time,\Ninter,\np,\island}$ be the Monte Carlo quantity constructed in ASIF-IR, conditional on $\vec{X}^{\IF}_{\time-1,\Ninter,\island}=\vec{x}^{\IF}_{\time-1,\Ninter,\island}$.
There is a constant $\BvConstant(\Unit,\Time,\Ninter)$ such that, for all $\efive^{}>0$ and $\vec{x}^{\IF}_{\time-1,\Ninter,\island}$, whenever the number of particles satisfies $J > \BvConstant(\Unit,\Time,\Ninter)/\efive^{4}$,
\begin{equation}
\nonumber
\hspace{-0.5mm}
\left|  
\, 
  \EMC\Big[ 
    \frac{1}{\Np} \sum_{\np=1}^{\Np} h(\vec{X}^{\GR}_{\time,\Ninter,\np,\island}) 
   \Big] \!
-  \E_g \! \big[h(\vec{X}_\time)\given \vec{X}_{\time-1}=\vec{x}^{\IF}_{\time-1,\Ninter,\island}\big] 
\right| \!
< \efive^{}.
\end{equation}
\end{assumptionB}
}

\newcommand\Bvi{
\begin{assumptionB}\label{B:XG_XA_ind}
For $1\leq \time \leq N$, the Monte Carlo random variable $X^A_{\time,i}$ is independent of $w^M_{\unit,\time,i,j}$ conditional on $X^A_{\time-1,i}$.
\end{assumptionB}
}


\newcommand\TheoremII{
\begin{theorem}  \label{thm:asif}
Let $\MC{\loglik}$ denote the Monte Carlo likelihood approximation constructed by ASIF-IR, or by ASIF since this is the special case of ASIF-IR with $\Ninter=1$.
Consider a limit with a growing number of islands, $\Island\to\infty$, and suppose assumptions~\ref{B1},~\ref{B1b}, \ref{B2}, \ref{B:temporal:mix}, \ref{B:girf} and~\ref{B:XG_XA_ind}. 
Suppose the number of particles $\Np$ exceeds the requirement for~\ref{B:girf}.
There are quantities $\el(\Unit,\Time)$ and $V(\Unit,\Time)$ with $|\el|<
Q^2\ethree + 2 Q^{2\Bsize}\big(\boundLemma\big)$
and $V<Q^{4\Bsize}\Unit^2\Time^2$ 
such that
\begin{equation}
\siOnly{\label{th:asif:lik:bound}}
\msOnly{\nonumber}
\Island^{1/2}
  \big[ 
    \MC{\loglik}-\loglik-\el\Unit\Time 
  \big]  
  \xrightarrow[\Island \rightarrow \infty]{d} 
  \normal\big[ 0,V \big].
\end{equation}
If additionally Assumption~\ref{B:unconditional:mix} holds, we obtain an improved rate of
\begin{equation}
\siOnly{\label{th:asif:lik:bound2}}
\msOnly{\nonumber}
V < Q^{4\Bsize}\Time\Unit
\big\{ c + \big(\boundOffDiagonal \big) \big(\Time\Unit-c\big)
\big\}
\end{equation}
\end{theorem}
}



%% for flow diagrams
\usepackage{tikz}
\usetikzlibrary{positioning}
\usetikzlibrary {arrows.meta}
\usetikzlibrary{shapes.geometric}

\begin{document}

\begin{frame}
  
\begin{center}
  {\Large\bf An iterated block particle filter for inference on coupled dynamic systems}

% ABSTRACT
% We consider inference for a collection of partially observed, stochastic, interacting, nonlinear dynamic processes. Each process is called a unit, and our primary motivation arises in biological metapopulation systems where a unit is a spatially distinct sub-population. Block particle filters are an effective tool for simulation-based likelihood evaluation for these systems, which are strongly dependent through time on a single unit and relatively weakly coupled between units. Iterated filtering algorithms can facilitate likelihood maximization for simulation-based filters. We introduce a new iterated block particle filter algorithm applicable to parameters that are either unit-specific or shared between units. We demonstrate this algorithm to carry out inference on a coupled epidemiological model for spatiotemporal COVID-19 case report data in 373 cities.


\vspace{2mm}

Edward Ionides\\
University of Michigan, Department of Statistics

\vspace{8mm}

IMS Asia Pacific Rim Meeting\\
Jan 5, 2024


\hspace{3mm}

Slides are at \url{https://ionides.github.io/talks/aprm24.pdf}

\vspace{8mm}

Joint work with
Patricia Ning, Jesse Wheeler, Kidus Asfaw, Jifan Li, Joonha Park, Aaron King, Mercedes Pascual

\end{center}

\end{frame}

\begin{frame}{Question to be addressed}

  \begin{enumerate}
  \item When can we carry out full-information likelihood-based inference on nonlinear non-Gaussian spatiotemporal partially observed Markov process (SpatPOMP) models? In particular, models for networks of interacting biological population dynamics.

        \vspace{2mm}

\item We introduce the {\bf iterated block particle filter}, currently the most powerful algorithm available in the \texttt{spatPomp} R package.

%  \vspace{2mm}

%  \item Bonus question: How do we know if our model is statistically adequate, or needs more work?
  
    \end{enumerate}
\end{frame}

\newcommand\challengeSep{\vspace{3mm}}

\begin{frame}{Inference challenges in population dynamics}

  \begin{enumerate}
\item Combining measurement noise and process noise.
\item Including covariates in mechanistically plausible ways.
\item  Continuous time models.
\item  Modeling and estimating interactions in coupled systems.
\item  Dealing with unobserved variables.
\item  \myemph{Modeling spatiotemporal dynamics}.
\item  Studying population dynamics via genetic sequence data.
  \end{enumerate}

  \vspace{4mm}
  
  1--5 are largely solved, from a methodological perspective.\\
  6 is our immediate topic.\\
  7 is exciting but not the focus of this talk.


  \vspace{4mm}

  \resizebox{12cm}{!}{Reviews: Bjornstad \& Grenfell ({\it Science}, 2001); Grenfell et al ({\it Science}, 2004)}  
  
\end{frame}

\begin{frame}{Desiderata}

  \begin{itemize}
    \item Consideration of arbitrary dynamic models. The limitations should be our scientific creativity and the information in the data.

      \vspace{2mm}
      
     Hence, \myemph{plug-and-play} methods which need a simulator from the model but not nice closed-form expressions for densities.

     \vspace{2mm}
     
    \item Statistically efficient inference, to extract all the information in the data.

      \vspace{2mm}
      
    Hence, \myemph{likelihood-based} methods.

      \end{itemize}
  \end{frame}

\begin{frame}{Fitting mechanistic models to time series}
  
    \vspace{8mm}

    \begin{itemize}
    \item  Iterated particle filtering via \texttt{mif2} in the R package \texttt{pomp} enables Masters-level statisticians to do plug-and-play likelihood-based inference:

      \vspace{2mm}

      \url{https://ionides.github.io/531w22/}

      \vspace{5mm}
      \item
      The science may be hard, but the statistics is becoming routine.

    % examples: COVID etc, bison in Yellowstone National Park, volatility in the stock market, gun purchases, human proximal small intestine pH

    \end{itemize}
\end{frame}

\begin{frame}{The curse of dimensionality}

  \bi
  \item
    Particle filter (PF) methods fail for high-dimensional systems. They scale exponentially badly.

    \vspace{2mm}
    
  \item Algorithms with improved scalability include:\\

    \vspace{1mm}
    
  {\bf
  Bagged filters (BF, IBF)\\
  Ensemble Kalman filter (EnKF, IEnKF)\\
  Guided intermediate resampling filter (GIRF, IGIRF)\\
  Block particle filter (BPF, IBPF)\\
  }

      \vspace{2mm}

\item Filters estimate latent states and evaluate the likelihood.

    \vspace{2mm}

  \item Iterated filters estimate parameters using stochastic parameter perturbations.

        \vspace{2mm}

\item These algorithms are all implemented in the \texttt{spatPomp} R package.
  
  \ei
  
\end{frame}



\begin{frame}{COVID in 375 cities in China, 10-23 January, 2020}

  \begin{itemize}
  \item Metapopulation data were used to infer the fraction of asymptomatic cases and their contagiousness
    (Li et al, {\it Science}, May 2020).
  \item SEIR (susceptible-exposed-infected-removed) model with asymptomatics, reporting delay, and coupling based on cell phone data.
  \item Li et al (2020) used iterated EnKF for inference.
  \item The time interval covers the initial China lockdown.
  \item Here, we summarize our re-analysis of this model and data, recently posted on arXiv.
%% see README for R code
%   \item Total cases 801, Wuhan 454, Chongqing 27, Beijing 26.
%   \item{} 259 cities with 0 cases, 99 with 1--5 cases, 17 with $>5$ cases.
%   \item{} 0 cases reported for all cities during Jan 10--15.

\end{itemize}     

\end{frame}

\begin{frame}
  %%{An SEAIR metapopulation model for COVID-19}
\begin{center}
%%%%% SEAIR diagram
\resizebox{11cm}{!}{
\begin{tikzpicture}[
  square/.style={rectangle, draw=black, minimum width=0.8cm, minimum height=0.8cm, rounded corners=.1cm, fill=blue!8},
  travel/.style={circle, draw=black, minimum width=0.85cm, minimum height=0.8cm, fill=green!8},
  report/.style={shape=regular polygon, regular polygon sides=8, draw, fill=red!8,minimum size=0.9cm,inner sep=0cm},
  bendy/.style={bend left=25},
  >/.style={shorten >=0.25mm}, % redefine arrow to stop short of node
  >/.tip={Stealth[length=1.5mm,width=1.5mm]} % redefine arrow style
]
\tikzset{>={}}; % this is needed to implement the arrow redefinition

\node (S) at (-0.5,0) [square] {S$_u$};
\node (E) at (2,0) [square] {E$_u$};
\node (A) at (4.5,0) [square] {A$_u$};
\node (I) at (4.5,-1.75) [square] {I$_u$};
\node (R) at (7,0) [square] {R$_u$};

\node (T1) at (-0.5,1.75) [travel] {T};
\node (T2) at (2,1.75) [travel] {T};
\node (T3) at (4.5,1.75) [travel] {T};

\node (Ca) at (4.5,-3.5) [report] {C$^a_u$};
\node (Cb) at (7,-3.5) [report] {C$^b_u$};
\node (C) at (9.5,-3.5) [report] {C$_u$};

\node (V1) at (2.5,-1.75) {};
\node (V2) at (3,-3.5) {};
\draw [->] (E) -- (Ca -| V2) -- (Ca);
\draw [->] (I -| V1) -- (I);

\draw [->] (S) -- (E);
\draw [->] (A) -- (R);

\draw [->] (Ca) -- (Cb);
\draw [->] (Cb) -- (C);

\draw [->] (I.east) -- (R);
\draw [->] (E) -- (A);

\draw [->, bendy] (S) to (T1);
\draw [->, bendy] (T1) to (S);

\draw [->, bendy] (E) to (T2);
\draw [->, bendy] (T2) to (E);

\draw [->, bendy] (A) to (T3);
\draw [->, bendy] (T3) to (A);

\end{tikzpicture}
}
\end{center}
\bi
\item 
  Reportably infectious individuals, $\mathrm{I}_u$ for city $u$, are included in the delayed reporting compartment, $\mathrm{C^a_u}$.
  \item 
An individual arriving at $\mathrm{C}_u$ is a case report for city $u$.
\item
  Individuals in $\mathrm{A}_u$ are not reportable and transmit at a reduced rate.
  \item 
Travel occurs to and from $\mathrm{T}$, based on 2018 data from Tencent.
%\item Many models have been proposed. We want a framework to implement and evaluate various scientific hypotheses.
\ei

\end{frame}

\begin{frame}
  \begin{center}
    \includegraphics[height=9.5cm]{covid/panel_plot-1.png}
    \end{center}
\end{frame}


\begin{frame}
 \includegraphics[width=12cm]{covid/filter_tests10.png}

  \begin{itemize}
\item BPF is the best available filter for this COVID model.
\item Similar results hold for other models (Ionides et al, {\it JASA}, 2023).
  \end{itemize}
  
\end{frame}

\begin{frame}{More on the block particle filter (BPF)}

\bi
\item BPF also worked quickly, easily and reliably on a measles metapopulation, as well as various toy benchmark problems (Ionides et al, {\it JASA}, 2023).

        \vspace{2mm}

\item BPF has theoretical support in some situations (Rebeschini \& Van Handel, {\it Annals of Applied Probability}, 2015).

        \vspace{2mm}

      \item This motivated us to develop an IBPF for parameter estimation.
        
        \vspace{2mm}

      \item IBPF has theoretical guarantees similar to BPF (Ning \& Ionides, {\it JMLR}, 2023).
        
        \vspace{2mm}

        
\item BPF was independently proposed as the ``factored particle filter'' by Ng et al (2002).

\ei

\end{frame}

\begin{frame}{Particle filter (PF)}

  \begin{columns}
    \begin{column}{0.48\linewidth}
      \begin{center}
      {\bf \textcolor{blue}{Evolutionary analogy}}

      \vspace{5mm}
      
      {\bf Mutation}

      $\downarrow$

      {\bf Fitness}

      $\downarrow$

      {\bf Natural selection}
      
      \end{center}
    \end{column}
     \begin{column}{0.48\linewidth}
      \begin{center}
      {\bf \textcolor{blue}{Particle filter algorithm}}

      \vspace{5mm}
      
      {\bf Predict: stochastic dynamics}

      $\downarrow$

      {\bf Measurement: weight}

      $\downarrow$

      {\bf Filter: resample}
      \end{center}
    \end{column}
  \end{columns}

  \vspace{15mm}
  
    \begin{myitemize}
  \item PF is an evolutionary algorithm with good mathematical properties: an unbiased likelihood estimate and consistent latent state distribution.
  \end{myitemize}

\end{frame}
  
\begin{frame}{Block particle filter (BPF)}

  \begin{myitemize}
    \item Blocks are a partition of the metapopulation units.
    \item For measles, we use each city as a block.
  \end{myitemize}
  
  \begin{columns}
    \begin{column}{0.48\linewidth}
      \begin{center}
      {\bf \textcolor{blue}{Evolutionary analogy}}

      \vspace{5mm}
      
      {\bf Mutation}

      $\downarrow$

      {\bf Fitness\\
      for each chromosome}

      $\downarrow$

      {\bf Natural selection\\
      for each chromosome}

      $\downarrow$

      {\bf Recombine chromosomes}
      
      \end{center}
    \end{column}
     \begin{column}{0.48\linewidth}
      \begin{center}
      {\bf \textcolor{blue}{Block particle filter}}

      \vspace{5mm}
      
      {\bf Predict: stochastic dynamics}

      $\downarrow$

      {\bf Measurement: weight\\
      for each block}

      $\downarrow$

      {\bf Filter: resample\\
      for each block}

      $\downarrow$

      {\bf Recombine blocks}
      \end{center}
    \end{column}
  \end{columns}

  \vspace{5mm}
  
    \begin{myitemize}
    \item Blocks are segments of the full state which can be reassorted between particles at the resampling step.

\end{myitemize}

\end{frame}


\begin{frame}{Comments on the Ensemble Kalman Filter (EnKF)}

  \begin{itemize}
  \item EnKF is more dependent on approximate Gaussianity than is sometimes supposed.

    \vspace{2mm}
    
  \item The Gaussian-inspired update rule is similar to the extended Kalman filter (EKF), which has largely been superseded by particle filter methods for low-dimensional nonlinear biological dynamics.

    \vspace{2mm}
    
\item Simple systems can defeat EnKF: the linear Gaussian update is helpless when data inform the conditional variance rather than the conditional mean.

  \vspace{2mm}
  
    \item Big systems need computationally tractable analysis. EnKF may sometimes be the best solution available, but be aware of its limitations.

  \end{itemize}

\end{frame}


\begin{frame}{An iterated block particle filter for parameter estimation}


  \begin{center}
    
  \includegraphics[trim={0 0 0 10mm},clip,width=9cm]{IBPF_workflow.pdf}


  \end{center}
  
\end{frame}

\begin{frame}{Practical inference using IBPF}
  \begin{enumerate}
  \item Monte Carlo adjusted profile likelihood \citep{ionides17profile} obtains confidence intervals that accommodate Monte Carlo error.

  \item Comparing the log-likelihood with an autoregressive model (or other simple statistical model) provides a check of model fit.

  \item Comparing the block log-likelihood against the benchmark provides insight into problematic units.

    \item Comparing the conditional log-likelihood for each observation against the benchmark helps to identify outliers. 
    
  \end{enumerate}
  
  \end{frame}

%% \begin{frame}{Technical issues with EnKF}

%%   \begin{itemize}
%%   \item EnKF is based on a continuous Gaussian approximation.
%%   \item Log-likelihoods with respect to counting measure are never positive, and cannot properly be compared with log-likelihoods corresponding to continuous densities.
%%     \item For data with many zeros, the unbounded EnKF likelihood can substantially bias the MLE.
%%     \item Adapting EnKF for count data is non-trivial (Katzfuss et al, {\it JASA}, 2019).
%% %    \item Li et al avoided the issue by using a ``heuristic observation error variance'' within EnKF. The corresponding ``log-likelihood'' is not the log-likelihood of any model. In particular, it cannot validly be compared againt simple statistical benchmarks.
%% %  \item Data and models will never be perfect. We want to reveal, rather than conceal, limitations.
%%   \end{itemize}
  
%%   \end{frame}

%% \begin{frame}{Future work}

%%   \newcommand\futuresep{\vspace{5mm}}
  
%%   \begin{myitemize}

%%   \item We anticipate continued progress in modeling and inference for nonlinear spatiotemporal partially observed stochastic dynamic systems.

%%     \futuresep
    
%%   \item As long as the \texttt{spatPomp} package facilitates data analysis, we expect continuing package development.

%% %    \futuresep

%%  %   \item Questions and comments?
    
%% \end{myitemize}

%% \end{frame}

\begin{frame}{Filtering $U$ units of a coupled measles SEIR model}

\vspace{-1mm}

\begin{center}
\includegraphics[width=10cm]{mscale_loglik_plot-1.pdf}


\end{center}

\vspace{-4mm}

Simulated data using a gravity model with geography, demography and transmssion parameters corresponding to UK pre-vaccination measles (Ionides et al, {JASA}, 2023).

%\end{center}

\end{frame}


\begin{frame}
\frametitle{Filtering $U$-dimensional correlated Brownian motion}

\vspace{-3mm}

\begin{center}
\includegraphics[width=10cm]{bm_alt_plot-1.pdf}

\vspace{-1mm}

$\cov\big(X_{\unit,\time}-X_{\unit,\time-1},X_{\altUnit,\time}-X_{\altUnit,\time-1}\big) \sim 0.4^{|\unit-\altUnit|}_{}$

\end{center}

\end{frame}

\begin{frame}
\frametitle{Filtering $U$ units of Lorenz 96 toy atmospheric model} 

\vspace{-3mm}

\begin{center}
\includegraphics[width=10cm]{lz_loglik_plot-1.pdf}

\vspace{-1mm}

$dX_{\unit}(t) = \big \{  X_{\unit-1}(t) \big(X_{\unit+1}(t) - X_{\unit-2}(t)\big) - X_{\unit}(t) + F \big\} dt + \sigma \, dB_{\unit}(t)$

\end{center}

\end{frame}


\nocite{asfaw23arxiv,bjornstad01,breto19,grenfell04,he10,katzfuss19,ionides21,ionides22,lee20,li20,ng02,ning23-ibpf,park20,rebeschini15,wheeler23}

\begin{frame}[allowframebreaks]
\frametitle{References}
\bibliographystyle{apalike}
\bibliography{bib}
\end{frame}

%%% extra material


\end{document}
